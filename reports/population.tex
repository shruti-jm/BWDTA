This section summarizes the literature review and analysis on the population distribution of BWDs in the Milky Way in terms of their mass, frequency, spatial distribution and other properties.

White dwarfs are remnant stars that sustain from collapse by electron degeneracy pressure. The upper limit to their mass is 1.4 M$_\odot$, given by the Chandrasekhar limit. 95\% of all stars become white dwarfs, and further, since two of these can become gravitationally bound by a common envelope phase to form a binary, a large number of such BWDs is also expected. These binaries can either be \textit{detached} (with no mass exchange) or \textit{interacting} (with tidal interaction and mass exchange). While interacting BWDs are easier to observe in EM, for our purpose as antenna for secondary GW sources, detached binaries are preferred due to it being easier to analyze gravitationally.

The first binary white dwarf (BWD) was observed in 1967 (cite) and the first detached BWD in 1988.

\begin{table}[h]
    \centering
    \begin{tabular}{c|c}
         &  \\
         & 
    \end{tabular}
    \caption{Modified from Table 1 of \cite{Lamberts2019}}
    \label{tab:lamberts}
\end{table}

\subsection{Population density and spatial distribution}
and Table 1 of \cite{Lamberts2019} 

\subsection{Mass distribution}
The upper limit to the mass of a single white dwarf being 1.4 M$_\odot$ 


\subsection{Frequency distribution}

\subsection{Peculiar velocities}